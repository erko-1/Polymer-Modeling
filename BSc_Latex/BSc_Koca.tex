% This is a sample LaTeX input file.  (Version of 12 August 2004.)
%
% A '%' character causes TeX to ignore all remaining text on the line,
% and is used for comments like this one.

\documentclass{article}      % Specifies the document class

\usepackage{graphicx}		 % Need this to include images
\usepackage{hyperref}
%% Define a HUGE 
\makeatletter
\newcommand\HUGE{\@setfontsize\Huge{50}{60}}
\makeatother

\begin{document}             % End of preamble and beginning of text.

 
%titlepage
\thispagestyle{empty}
\begin{center}
\begin{minipage}{0.9\linewidth}
\flushright
	      		 
	%University logo
    \includegraphics[width=0.5\linewidth]{univie.jpg}\par
    \vspace{1.5cm}
\centering 	
    % Title
	{\scshape{\HUGE Bachelorarbeit\par}}
	\vspace{1cm}
	%Thesis title
    {\scshape{\Large Computational Probability Analysis Of Polymers \par}}
    \vspace{2cm}
    
  
 verfasst von  \linebreak
 {\Large Eray Koca\par}
 	\vspace{1.5cm}
angestrebter akademischer Grad\linebreak
 {\Large Bachelor of Science (BSc)\par}
	\vspace{1.5cm}

\flushleft
	

\begin{tabular}{ll}
Wien, 2025	\linebreak
\vspace{1cm}&   \\
  Studienkennzahl lt. Studienblatt: & UA 033 676 \vspace{0.3cm} \\ 
  Fachrichtung: & Bachelorstudium Physik \vspace{0.3cm} \\
  Betreut von: &  Univ. -Prof. Dipl. -Ing Dr. Christos Likos\\
 \end{tabular}


    
    
\end{minipage}
\end{center}
\clearpage

\newpage

\begin{abstract}
This thesis focuses on computational polymer topology analysis. The study initially explores the behavior of linear polymers through simulations, progressing to ring polymers and finally concentrating on double-ring polymers. The primary goal is to gather data via computational simulations and analyze the probability distribution of the geometrical behavior of double-ring polymers.

In addition to advancing our understanding of polymer topology, this study aims to offer valuable computational insights, that encourage further experimental investigations to validate and build upon theoretical findings, ultimately enhancing the broader understanding of polymers.

Computational polymer topology analysis investigates polymer structure and properties at the molecular level using various methods and potentials. One crucial method is Monte Carlo simulation, which employs random sampling to explore potential polymer configurations. The Metropolis criterion, also known as the Boltzmann acceptance criterion, determines whether proposed movements are accepted based on the energy change, following the Boltzmann distribution.

Specific potentials model interactions within polymer structures. The FENE potential (Finitely Extensible Nonlinear Elastic) describes elastic stretching of polymer bonds and prevents unphysical elongations by applying a strong restoring force beyond a certain length limit. The WCA potential (Weeks-Chandler-Andersen), a truncated Lennard-Jones potential, describes short-range repulsive interactions and ensures non-overlapping monomers within a polymer chain, resulting in self-avoiding polymers.

The study defines two vectors originating from the midpoint between the rings to their centers of mass. By applying the Boltzmann distribution and statistically screening double-ring movements, a correlation factor can be determined.

These methods and potentials are employed to analyze polymer topology, enabling investigation into configurations, probabilities, and entanglement structures.Furthermore on this will be explained later. 


\end{abstract}

\newpage

\tableofcontents

\newpage

\section{Introduction}

Understanding polymer topology is crucial for designing materials with tailored properties, as different structures influence mechanical strength, elasticity, thermal behavior, and chemical resistance. Beyond material science, polymer topology plays a key role in biological systems, such as the structural organization of DNA.

Over the years, significant advancements have deepened our understanding of polymer topology. Early developments, including Staudinger’s macromolecular theory and Flory and Huggins’ statistical models, laid the foundation for studying polymer structures. Later, De Gennes introduced reptation theory to describe polymer dynamics, while knot theory provided a framework for analysing loops and entanglements. More recently, advances in supramolecular chemistry and controlled polymer synthesis have enabled precise topological control, with applications in nanotechnology and biomaterials (Flory, 1953; De Gennes, 1979).

Among various polymer architectures, ring polymers have drawn significant interest due to their unique topological constraints, which affect their physical behavior. Unlike linear polymers, ring polymers lack free ends, leading to distinct dynamic and mechanical properties. However, more complex topologies, such as "self-avoiding double-ring polymers", remain relatively unexplored, despite their potential for novel material and biological applications.

Previous studies have investigated the topological properties of ring polymers under different conditions. Christos N. Likos has made key contributions to understanding their threa    ding effects, hydrodynamics, and coarse-grained modeling, providing a foundation for studying more intricate topologies such as the double-ring (Likos, 2001).

Recent research by Schanek, Smrek, Likos, and Zöttl (2024) explores the behavior of supercoiled ring polymers under shear flow, offering insights into their complex conformational dynamics. Their study combines simulations and theoretical modelling to examine how shear forces influence polymer deformation, relaxation, and entanglement. They demonstrate that supercoiling significantly alters a polymer’s response to flow, leading to distinct stretching and tumbling regimes. These findings contribute to understanding the rheology of topologically constrained polymers and provide a framework for investigating more intricate architectures, such as the double-ring topology examined in this thesis (Schanek/Smrek/Likos/Zöttl, 2024). Their work sparked the interest for this work. 

This thesis aims to conduct computational simulations to analyse the topological behavior of double-ring polymers. With simulations it examines their probability distributions and energy development and provide information on the topological characteristics. By doing so, this research advances the understanding of polymer topology and its implications.

\newpage
  
\subsection{Polymers} 

To define a polymer, the first step is to identify a monomer, the fundamental building block. Monomers link together over covalent bonds to form polymer chains. One key characteristic used to describe a polymer is the degree of polymerization N, which represents the number of monomers in a polymer chain. This value is expressed as $N = \frac{M}{M_{mon}}$, where M being the number-average molar mass of the polymer, and $M_{mon}$ is the molecular weight of a single monomer. Since polymerization is a statistical process, the degree of polymerization is typically given as an average.


Each polymer consists of a primary chain, known as the \textbf{backbone}, which provides the fundamental structure of the polymer. This microstructure remains unchanged unless covalent bonds are broken. The length of the backbone, typically measured by the number of monomers it contains, is a key factor in describing the polymer chain.

Additionally, \textbf{isomers} are molecules that share the same molecular formula but differ in the arrangement or position of their atoms. Isomers can be classified into structural isomers, that are molecules that have different connectivity between atoms and stereoisomers, that are molecules that have the same connectivity but different spatial orientations.

Furthermore, to determine the mass of a single molecule M, the following relation can be used:

\[
M = \frac{M_{mol}}{N_A}
\]

where $M_{mol}$ is the molar mass of the molecule, and $N_A$ is Avogadro's number $6.02 \cdot 10^{23} mol^{-1}$.





\subsection{Modelling}
ideal polymers
real polymer

\subsubsection{Monte Carlo Simulation}
book summary

\subsubsection{WCA and FENE}
self avoiding and explanation of potentials

\subsubsection{Metropolis Criterion}
statistical explanation

\subsection{Double-Ring Polymers}
my own work

\subsection{Conclusion}



\pagebreak

\bibliographystyle{acm}
\bibliography{BSc_Latex_Template}

\end{document}